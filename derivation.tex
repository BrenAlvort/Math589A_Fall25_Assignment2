\documentclass{article}
\usepackage{amsmath,amssymb,amsfonts,fullpage}
\usepackage{microtype}
\title{An algorithm outline to solve a general linear system}
\author{Marek Rychlik and ChatGPT-5}
\begin{document}
\maketitle
Assume \(PA=LU\) is a row‑permuted LU factorization of \(A\) (so \(A=P^{T}LU\)). Let \(b\) be given and consider \(Ax=b\). Multiplying by \(P\) gives
\[
LUx=Pb.
\]
Introduce \(y:=Ux\). Then the system is equivalent to the triangular systems
\[
Ly=Pb,\qquad Ux=y.
\]

The columns of \(U\) fall into two classes: pivot (basic) columns and nonpivot (free) columns. Let the pivot column indices be \(B\subset\{1,\dots,n\}\) and the complement (free) column indices be \(F=\{1,\dots,n\}\setminus B\). If \(|B|=r\), then the basic variables are the components of \(x\) indexed by \(B\) and the free variables are those indexed by \(F\).

Important remark: the pivot columns of \(U\) are not, in general, the leftmost columns of \(U\). In other words, \(U_B\) (the matrix formed from the pivot columns) is usually not equal to the left block of \(U\) unless a suitable column reordering has been applied. To work with a block partitioned form of \(U\) one therefore needs to perform a column permutation (or equivalently select columns by index). We now explain this in detail.

Define a permutation matrix \(Q\in\mathbb{R}^{n\times n}\) that reorders the standard basis so that the pivot column indices \(B\) are moved to the first \(r\) positions and the free indices \(F\) occupy the remaining \(n-r\) positions. Equivalently,
\[
UQ = \bigl[\,U(:,B)\ \; U(:,F)\,\bigr] =: \begin{pmatrix}U_B & U_F\end{pmatrix},
\]
where \(U(:,B)\) denotes the submatrix of \(U\) formed by the columns indexed by \(B\). (If one prefers not to form an explicit \(Q\), one may work directly with the column‑selection operators \(U_B=U(:,B)\) and \(U_F=U(:,F)\), but introducing \(Q\) makes the block formulas below concise.)

Correspondingly permute the unknowns via
\[
x' := Q^{T}x,
\]
so that \(x' = \begin{pmatrix}x'_B\\ x'_F\end{pmatrix}\) with \(x'_B\in\mathbb{R}^{r}\) the basic components and \(x'_F\in\mathbb{R}^{\,n-r}\) the free components in the reordered variable vector. Then
\[
Ux = U(QQ^{T})x = (UQ)x' = \begin{pmatrix}U_B & U_F\end{pmatrix}\begin{pmatrix}x'_B\\[2pt] x'_F\end{pmatrix}.
\]

As usual, restrict attention to the pivot rows (the rows that contain the leading entries of the selected pivot columns). After discarding any all‑zero trailing rows of \(U\) we obtain an \(r\times r\) upper triangular invertible submatrix formed by the pivot rows and pivot columns; this is the invertible \(U_B\) referred to above (viewed as \(r\times r\) once restricted to pivot rows) and \(U_F\) is the corresponding \(r\times(n-r)\) block of free columns restricted to the pivot rows. With these conventions the equation \(Ux=y\) becomes, in the permuted variables,
\[
U_B x'_B + U_F x'_F = y_{\mathrm{pivot}},
\]
where \(y_{\mathrm{pivot}}\) denotes the entries of \(y\) corresponding to the pivot rows (equivalently, \(y_{\mathrm{pivot}}\) is the leading \(r\) entries of \(y\) after discarding any zero trailing rows of \(U\)). Solving for the basic variables gives
\[
x'_B = U_B^{-1}\bigl(y_{\mathrm{pivot}} - U_F x'_F\bigr).
\]
Hence the general solution in the permuted coordinates is
\[
x'=\begin{pmatrix}x'_B\\[2pt] x'_F\end{pmatrix}
=\begin{pmatrix}U_B^{-1}y_{\mathrm{pivot}}\\[2pt] 0\end{pmatrix}
+\begin{pmatrix}-U_B^{-1}U_F\\[2pt] I_{\,n-r}\end{pmatrix}x'_F,
\qquad x'_F\in\mathbb{R}^{\,n-r}\ \text{arbitrary}.
\]

Returning to the original variable ordering \(x = Qx'\), the particular part and the null‑space part of the solution in the original coordinates are obtained by applying \(Q\) to the two blocks above. Thus the null‑space mapping \(N\) (which maps free parameters to full solutions in the original ordering) and the particular vector \(c\) are
\[
N \;=\; Q
\begin{pmatrix}-U_B^{-1}U_F\\[3pt] I_{\,n-r}\end{pmatrix}
\in\mathbb{R}^{\,n\times(n-r)},\qquad
c \;=\; Q\begin{pmatrix}U_B^{-1}y_{\mathrm{pivot}}\\[3pt]0\end{pmatrix}.
\]

Finally, using invertibility of \(L\) we express \(y\) in terms of \(b\):
\[
y=L^{-1}Pb,
\]
and \(y_{\mathrm{pivot}}\) is the restriction of this vector to the pivot rows. Substituting gives the particular vector for the original system \(Ax=b\):
\[
c \;=\; Q\begin{pmatrix}U_B^{-1}L^{-1}Pb_{\mathrm{pivot}}\\[3pt]0\end{pmatrix}
= Q\begin{pmatrix}U_B^{-1}L^{-1}Pb\\[3pt]0\end{pmatrix},
\]
where the latter equality indicates that \(U_B^{-1}\) is applied to the entries of \(L^{-1}Pb\) corresponding to the pivot rows (zero entries in trailing rows of \(U\) play no role).

Therefore the full solution of \(Ax=b\) in the original variable ordering is
\[
x = N x_F + c,
\]
with
\[
N = Q\begin{pmatrix}-U_B^{-1}U_F\\[3pt] I_{\,n-r}\end{pmatrix},\qquad
c = Q\begin{pmatrix}U_B^{-1}L^{-1}Pb\\[3pt]0\end{pmatrix},
\]
and \(x_F\in\mathbb{R}^{\,n-r}\) arbitrary. Note that \(N\) depends only on the LU factor (and the pivot pattern encoded by \(Q\)), not on \(b\).

In practice one may implement these steps either by (i) forming the column‑permutation \(Q\) that brings pivot columns to the front and working with the left block \(U_B\), or (ii) by working directly with indexed column selections \(U(:,B)\) and \(U(:,F)\) and mapping the components of \(x\) between the original and reduced orderings as required. The algebra above makes explicit how the required column operations (reordering or selection) enter into the construction of \(U_B\), \(U_F\), \(N\) and \(c\).

\end{document}
